\documentclass[12pt,a4paper]{article}
\usepackage{kotex}
\usepackage{amsmath,amssymb,amsthm}
\usepackage{graphicx}
\usepackage{natbib}
\usepackage{hyperref}
\usepackage{geometry}
\usepackage{setspace}
\usepackage{indentfirst}

\geometry{left=25mm,right=25mm,top=30mm,bottom=30mm}
\setstretch{1.5}

\newtheorem{theorem}{정리}[section]
\newtheorem{proposition}[theorem]{명제}
\newtheorem{lemma}[theorem]{보조정리}
\newtheorem{corollary}[theorem]{따름정리}
\theoremstyle{definition}
\newtheorem{definition}[theorem]{정의}
\newtheorem{axiom}[theorem]{공리}
\theoremstyle{remark}
\newtheorem{remark}[theorem]{비고}
\newtheorem{example}[theorem]{예제}

\title{\textbf{중력 축소 가설:\\수축 기반 신우주론}}
\author{박주현\\[0.5em]\small genesos@gmail.com}
\date{}

\begin{document}

\maketitle

\begin{abstract}
일반상대성이론은 중력을 시공간의 곡률로 성공적으로 설명한다. 하지만 곡률이라는 기하학적 언어는 개념적으로 어렵다. 본 연구에서는 곡률 대신 ``국소 시공간 수축''이라는 개념으로 일반상대성이론을 재해석한다. 이 관점에서 각 질량은 자신의 바로 주변 시공간만을 국소적으로 수축시킨다. 원격 작용은 일어나지 않는다. 질량이 있든 없든 모든 입자는 이렇게 변형된 기하학 안에서 측지선을 따라 움직인다.

본 연구는 이 ``수축 관점''이 표준 슈바르츠실트 해와 수학적으로 완전히 동등함을 증명한다. 일반상대성이론의 모든 고전적 예측을 그대로 재현한다. 핵심적인 개념적 장점은 잘 정립된 초기값 정식화를 통해 원격 작용이 없음을 명시적으로 보여준다는 점이다. 질량은 오직 국소적인 시공간 기하학만 바꾼다. 중력 효과는 빛의 속도로 인과적으로 전파된다.

본 연구는 사영좌표계 형식과의 관계를 통해 수축률의 좌표 의존성을 엄밀하게 다룬다. 고유거리와 좌표거리에 대한 신중한 분석으로 ``수축''이라는 용어를 명확히 한다. 다체계 처리 방법을 제시하고, 텔레평행 중력과의 명시적인 수학적 연결을 보여준다.

본 연구는 주로 개념적 명확성을 제공하는 교육적 재정식화다. 그러나 흥미롭게도, 이 수축 관점은 자연스럽게 우주론적 확장의 가능성을 제안한다. 정적 시공간에서는 일반상대성이론과 수학적으로 동등하지만, 우주론적 규모에서는 황금 지수 스케일링, 거리 이중성 관계 위반, 허블 텐션 해결 등 잠재적인 관측 신호를 포함한다. 이런 확장은 매우 추측적이며 향후 엄밀한 검증이 필요하다.
\end{abstract}

\section{서론}

\subsection{역사적 배경: 원격 작용에서 휘어진 시공간으로}

원격 작용 문제는 뉴턴이 만유인력을 정식화한 이후 물리학자들을 괴롭혀 왔다. 뉴턴 자신도 ``중력이 물질에 내재하여 한 물체가 진공을 통해 떨어진 다른 물체에 작용할 수 있다''는 생각에 불편함을 느꼈다 (Newton, 1692/1693, Bentley에게 보낸 편지). 뉴턴 역학이 놀라운 경험적 성공을 거두었음에도, 이런 개념적 긴장은 2세기 넘게 이어졌다.

아인슈타인의 일반상대성이론(Einstein, 1916)은 혁명적인 해결책을 제시했다. 중력은 거리를 두고 작용하는 힘이 아니다. 시공간의 곡률로 나타난다. 휠러의 유명한 격언처럼, ``물질은 시공간에게 어떻게 휘라고 말하고, 시공간은 물질에게 어떻게 움직이라고 말한다'' (Misner, Thorne, \& Wheeler, 1973). 이 기하학적 패러다임은 놀라운 성공을 거두었다. 근일점 세차(Clemence, 1947)부터 중력파 검출(Abbott et al., 2016)까지 수많은 실험이 이를 확인했다.

그러나 ``곡률''이라는 언어 자체가 어려움을 준다. 리만 기하학으로 수학적으로는 정밀하지만, 많은 물리학자에게 개념적으로 추상적이다. 이런 어려움은 대안적 정식화가 새로운 관점을 제공할 수 있음을 시사한다.

\subsection{동기: 더 직관적인 기하학적 언어를 향해}

저의 핵심 동기는 원격 작용을 제거한다는 것이 무엇을 의미하는지 다시 생각하는 데서 출발한다. 아인슈타인은 중력 상호작용을 기하학적 변형으로 바꿀 수 있음을 보였다. 하지만 ``곡률''이라는 특정 언어가 이 기하학적 통찰을 표현하는 유일한---또는 최선의---방법인가?

그렇지 않다고 본다. 같은 풍경을 다른 지도 투영법으로 나타낼 수 있듯이, 같은 기하학적 구조를 다른 수학적 언어로 기술할 수 있다. 문제는 어떤 언어가 물리적 직관을 가장 잘 포착하는가이다.

\textbf{핵심 질문:} 질량들이 서로 공간을 가로질러 상호작용하지 않고, 단지 자신의 바로 주변만 바꾼다는 것을 어떻게 가장 명확하게 표현할 수 있는가?

\textbf{저의 답:} 각 질량은 주변에 국소적인 시공간 수축---거리와 시간 측정의 재척도화---을 만든다. 중요한 점은, 이 수축이 개별 입자가 아닌 계량 자체에 작용한다는 것이다. 모든 물리 과정이 거리와 시간 간격을 정의하는 데 계량에 의존하므로, 모든 물질과 에너지가 수축에 똑같이 반응한다.

이 수축은 몇 가지 핵심 속성을 갖는다:
\begin{itemize}
\item \textbf{국소성}: 질량이 있는 영역에만 영향을 준다
\item \textbf{비등방성}: 방사, 접선, 시간 방향에서 다르다
\item \textbf{독립성}: 각 질량은 다른 질량과 무관하게 독립적으로 시공간을 수축시킨다
\item \textbf{보편성}: 모든 형태의 물질과 에너지에 똑같이 영향을 준다
\end{itemize}

보편성은 일반상대성이론의 등가원리에서 따라온다. 새로운 공리가 아니다. 이 정식화가 일반상대성이론과 수학적으로 동등하므로, 이 성질을 물려받는다. 대상이 재척도화된 기하학에서 측지선을 따르는 이유는 그들이 기본 시공간 계량에서 측지선을 따르기 때문이다.

\subsection{형식적 정의: ``원격 작용''이란 무엇인가?}

이 주장을 엄밀하게 정당화하려면, 먼저 장 이론 맥락에서 원격 작용이 무엇을 의미하는지 정확히 정의해야 한다.

\begin{definition}[원격 작용]
물리 이론이 시공간 점 $p$에서의 물질 상태나 동역학이 공간적으로 분리된 점 $q$에서의 물질 상태에 명시적으로 의존할 때만 원격 작용을 나타낸다. 여기서 간격은 계량 부호 $(-,+,+,+)$에서 $(p-q)^2 = g_{\mu\nu}(x^\mu_p - x^\mu_q)(x^\nu_p - x^\nu_q) > 0$을 만족한다.
\end{definition}

\begin{definition}[장 이론에서의 국소성]
이론이 점 $p$에서의 동역학이 다음에만 의존할 때 국소적이다:
\begin{enumerate}
\item $p$에서의 장 값
\item $p$에서 평가된 장의 도함수
\item $p$의 과거 빛원뿔 내의 점에서의 장 값
\end{enumerate}
\end{definition}

아인슈타인의 장 방정식으로 정식화된 일반상대성이론은 이런 의미에서 국소적이다:

$$G_{\mu\nu}(x) = \frac{8\pi G}{c^4} T_{\mu\nu}(x)$$

왼쪽(아인슈타인 텐서)은 점 $x$에서 계량 $g_{\mu\nu}$와 그 1차, 2차 도함수에만 의존한다. 오른쪽(에너지-운동량 텐서)은 점 $x$에서 물질장에만 의존한다.

그런데 ``점 A의 질량이 멀리 떨어진 점 B의 입자 운동에 어떻게 영향을 주는가?''라고 물으면 개념적 혼란이 생긴다. 뉴턴 중력에서는 비국소적 답을 갖는다. 일반상대성이론에서 답은 국소적이지만 미묘하다. A의 질량은 바로 주변 계량을 바꾼다. 이 변화는 인과적으로(속도 $c$로) 바깥으로 전파된다. 결국 B의 기하학에 영향을 준다. B의 입자는 B에서의 국소 계량에만 반응한다.

\begin{definition}[인과적 전파]
장 이론이 교란을 시간꼴 또는 영꼴 세계선을 따라(빛원뿔 내부 또는 위에) 전파하고, 공간꼴 세계선을 따라서는 결코 전파하지 않을 때 인과적 전파를 나타낸다.
\end{definition}

\textbf{핵심 질문:} 표준 일반상대성이론이 수학적으로 국소적이지만, 이 국소성을 더 투명하게 만드는 방식으로 정식화할 수 있는가? 본 연구에서는 ``국소 수축'' 언어가 이 목표를 달성한다고 본다.

\subsection{기존 대안 정식화와의 관계}

이 접근은 새로운 통찰을 얻기 위해 일반상대성이론을 재정식화하는 풍부한 전통에 합류한다:

\textbf{텔레평행 중력}(Weitzenböck, 1923; Aldrovandi \& Pereira, 2013)은 곡률을 중력의 기하학적 기술자로 비틀림으로 바꾼다. 고전 수준에서 일반상대성이론과 수학적으로 동등하지만, 다른 관점을 제공한다.

\textbf{형태 동역학}(Barbour, 2012)은 국소 등각 불변성을 공간 미분동형사상 불변성으로 교환하여 일반상대성이론을 재정식화한다.

\textbf{광학 기하학}(Abramowicz \& Lasota, 1997)은 빛 전파 관점에서 시공간을 기술하며, 직관적 시각화 도구를 제공한다.

각 재정식화는 뚜렷한 장점을 제공한다. 이 ``수축'' 틀은 개념적 명확성과 원격 작용의 명시적 제거를 강조한다. 표준 일반상대성이론과 완전한 수학적 동등성을 유지한다.

\subsection{본 연구의 목표와 범위}

본 논문의 목표:

\begin{enumerate}
\item ``국소 시공간 수축'' 개념을 수학적으로 정밀하게 \textbf{정의}
\item 일반상대성이론의 슈바르츠실트 해와의 동등성 \textbf{증명}
\item 수축 언어가 곡률 언어보다 국소성을 더 명시적으로 만드는 방법 \textbf{제시}
\item 수축률의 좌표 의존성 \textbf{다루기}
\item ``수축'' 용어 \textbf{명확히 하기}
\item 중력 현상(렌즈효과, 시간 팽창, 적색편이)을 수축 언어로 \textbf{설명}
\item 관련 재정식화, 특히 텔레평행 중력과 \textbf{비교}
\item 다체계 처리 \textbf{발전시키기}
\item 이 재정식화의 교육적 가치 \textbf{평가}
\item 우주론으로의 추측적 확장 가능성 \textbf{탐구}
\end{enumerate}

\textbf{범위와 한계:} 주된 초점은 구형 대칭, 정적 시공간(슈바르츠실트 해)에 있다. 여기서 수축 정식화가 일반상대성이론과 수학적으로 완전히 동등함을 엄밀하게 증명한다. 우주론적 확장(7절)은 추측적이며, 향후 이론적 발전과 관측 검증이 필요하다. 회전하는 블랙홀과 완전한 동역학 이론도 향후 연구로 남긴다.

\subsection{논문의 구조}

2절에서는 개념적 틀을 발전시킨다. 3절에서는 수학적 정식화를 제공하고 슈바르츠실트 계량과의 동등성을 증명한다. 4절에서는 중력파를 간략히 논의한다. 5절에서는 이 접근을 텔레평행 중력과 비교한다. 6절에서는 교육적 가치를 검토한다. 7절에서는 우주론으로의 추측적 확장 가능성을 탐구한다. 8절에서 결론을 맺는다.

\section{개념적 틀}

\subsection{표준 관점: 곡률로서의 중력}

아인슈타인의 일반상대성이론은 중력을 계량 텐서 $g_{\mu\nu}$로 인코딩된 시공간 곡률로 설명하며, 물질은 측지선을 따른다. 이것이 원격 작용을 성공적으로 제거하지만, 수학적 언어---리만 텐서, 리치 텐서, 측지선 편차---는 상당한 기하학적 추상화를 요구한다.

\subsection{수축 관점: 대안적 기하학적 언어}

``곡률''보다는 ``국소 수축'' 언어로 같은 기하학적 구조를 기술한다. 핵심 공리는:

\begin{axiom}[국소 수정]
한 점의 에너지-운동량은 그 점에서 시공간 계량을 국소적으로 바꾼다. 이 변화를 거리와 시간 측정의 재척도화로 기술한다.
\end{axiom}

\begin{axiom}[비등방성 수축]
수축은 방향에 따라 다르다:
\begin{itemize}
\item \textbf{방사 방향}: 공간 수축
\item \textbf{접선 방향}: 변화 없음
\item \textbf{시간 방향}: 시간 측정 변화
\end{itemize}
\end{axiom}

\begin{axiom}[측지선 운동]
모든 입자는 수정된 시공간 기하학에서 측지선을 따른다.
\end{axiom}

\begin{axiom}[독립성과 국소성]
각 질량은 주변에서 시공간을 독립적으로 바꾼다. 분리된 질량 사이에 직접적인 상호작용은 없다.
\end{axiom}

\begin{axiom}[보편성]
수축은 모든 물리 과정에 똑같이 영향을 준다. 이는 등가원리에서 물려받는다.
\end{axiom}

\subsection{엄밀한 정당화: 왜 수축이 국소성을 더 명시적으로 만드는가}

\begin{theorem}[초기값 정식화와 국소성]
일반상대성이론에서, 코시 표면 $\Sigma$의 초기값 문제는 시공간의 미래 진화를 유일하게 결정한다.
\end{theorem}

\begin{proof}[증명 개요]
이는 아인슈타인 방정식의 쌍곡선적 성질에서 따라온다(Wald, 1984). 중력 교란은 속도 $c$로 전파되며, 결코 더 빠르지 않다.
\end{proof}

\begin{theorem}[수축 정식화]
수축 틀에서, 초기 수축률 함수 $S_i(\vec{x}, t_0)$와 그 시간 도함수가 주어지면, 미래 수축률은 파동 방정식으로 유일하게 결정된다:

$$\Box S_i = \mathcal{F}[S_i, \partial S_i, T_{\mu\nu}]$$

여기서 $\Box$는 달랑베르시안 연산자다.
\end{theorem}

\begin{proof}[물리적 해석]
수축 변화는 파동으로 속도 $c$로 전파된다.
\end{proof}

\begin{proposition}[슈바르츠실트에서의 명시적 국소성]
정적 질량 $M$에 대해, 반지름 $r$에서의 수축률은:

$$S_r(r) = \sqrt{1 - \frac{r_s}{r}}, \quad r_s = \frac{2GM}{c^2}$$

이는 방사 좌표 $r$에만 의존하며, 먼 질량이나 과거 빛원뿔 밖 정보에는 의존하지 않는다.
\end{proposition}

\textbf{곡률 정식화와의 핵심 차이:}

두 정식화가 수학적으로 동등하지만, 개념적 강조에서 근본적으로 다르다:

\begin{enumerate}
\item \textbf{곡률 접근}: 256개 성분(4차원에서 20개 독립)을 가진 리만 텐서 $R^\rho_{\sigma\mu\nu}$ 계산 필요, 계량의 2차 도함수 포함. 물리적 의미는 측지선 편차 계산 후에야 나타남.

\item \textbf{수축 접근}: ``거리 측정은 인자 $S_r(r) = \sqrt{1 - r_s/r}$로 재척도화된다''고 직접 진술. 물리적 내용이 즉각적이며 계산 부담이 최소.
\end{enumerate}

\textbf{계산상 이점:} 중력 시간 팽창 예측을 위해, 곡률 정식화는: (1) 크리스토펠 기호 $\Gamma^\mu_{\nu\lambda}$ 계산, (2) 측지선 방정식 풀기, (3) 고유시간 해석이 필요. 수축 정식화: 계량에서 $S_t(r)$을 직접 읽음.

\textbf{교육적 이점:} 학생들은 ``질량 근처에서 공간이 압축된다''를 ``리만 텐서 성분 $R^r_{trt}$가 조석 가속도를 측정한다''보다 쉽게 이해한다.

\subsection{곡률 대 수축: 같은 영역의 다른 지도}

수학적 구조---계량 텐서 $g_{\mu\nu}$---수준에서 곡률과 수축은 동일하다. 슈바르츠실트 계량은 어느 관점에서든 유도할 수 있다.

그러나 \textbf{정식화들은 무엇을 명시적으로 만드는가에서 차이가 있다:}

\textbf{곡률 언어가 명시적으로 만드는 것:}
\begin{itemize}
\item 조석력(측지선 편차 방정식)
\item 좌표 독립적 기하학적 양(리만 텐서)
\item 전역 위상적 성질
\end{itemize}

\textbf{수축 언어가 명시적으로 만드는 것:}
\begin{itemize}
\item \textbf{명시적 국소성}: 각 $S_i(r)$은 국소 $r$에만 의존, 먼 원천에 의존하지 않음
\item \textbf{측정 재척도화}: 관측 가능한 양(시계 속도, 자 길이)과 직접 연결
\item \textbf{인과적 전파}: $S_i$의 변화는 속도 $c$로 파동 방정식을 만족
\item \textbf{힘의 부재}: ``$F = ma$'' 틀이 불필요---기하학 자체가 변형됨
\end{itemize}

\textbf{결정적 차이:}

표준 교과서에서 일반상대성이론의 국소성은 아인슈타인 방정식의 쌍곡선적 성질에 대한 정교한 분석으로 증명된다. 반면, 수축 정식화는 구성에 의해 국소성을 투명하게 만든다: 어떤 점에서 $S_i(\mathbf{x}, t)$는 그 점의 과거 빛원뿔 내 응력-에너지 텐서에만 의존한다.

이는 단순한 표기법이 아니다---``국소성이 보존됨을 증명''에서 ``국소성을 기초에 구축''으로의 전환을 나타낸다.

\textbf{우주론을 위한 고유한 통찰:}

수축 틀은 곡률 정식화가 가리는 질문을 자연스럽게 제기한다: 시공간 변형이 전파된다면, 전파 속도가 정확히 $c$와 같아야 하는가, 아니면 다를 수 있는가? 이 질문은 표준 일반상대성이론($c$가 계량 구조에 내장됨)에서는 물리적으로 무의미하지만 수축 언어($S_i$가 동역학적 장)에서는 자연스럽다. 이는 우주론적 긴장을 다루는 잠재적 확장을 연다(7.3절 참조).

\textbf{일반상대성이론과의 근본적 차이:}

고전 수준에서 수학적으로 동등하지만, 수축과 곡률 정식화는 아인슈타인의 원래 이론을 넘어 확장할 때 분기하는 다른 물리적 그림을 제시한다:

\textbf{표준 일반상대성이론 관점:}
\begin{itemize}
\item 중력 = 기하학적 성질(시공간 곡률)
\item 정보가 시공간을 통해 속도 $c$로 전파
\item 중력 효과는 원천에서 관측자로의 인과적 전파 필요
\item 암흑에너지 = 우주상수 $\Lambda$ (시공간의 정적 성질)
\end{itemize}

\textbf{수축 관점:}
\begin{itemize}
\item 중력 = 국소 과정(각 점에서 독립적 재척도화)
\item 정보 전파 불필요---각 점이 국소 물질 밀도에 기반하여 독립적으로 ``결정''
\item 중력 효과는 전달된 신호가 아닌 동기화된 국소 작용에서 나타남
\item 암흑에너지는 잔여 수축 효과일 수 있음(동역학적)
\end{itemize}

\textbf{핵심 질문:}

수축이 진정으로 국소적이고 독립적이라면, 왜 모든 점이 정확히 조정된 속도로 수축해야 하는가? 표준 일반상대성이론은 완벽한 조정(빛의 속도로 매개됨)을 요구한다. 하지만 각 점이 국소 물질 밀도에만 기반하여 독립적으로 작용한다면, ``조정''은 정확하기보다는 근사적일 수 있다.

이 개념적 구별은 우주론적 스케일로 확장할 때 물리적으로 의미를 갖는다: 수축 그림은 $c$와 다를 수 있는 수축 전파 속도를 자연스럽게 허용하는 반면, 표준 일반상대성이론은 $c$를 기하학적 기초에 내장한다. 자연이 이 가능성을 활용하는지는 열린 질문으로 남는다(7.3절 참조).

\section{수학적 정식화}

\subsection{수축률 함수의 정의}

구형 대칭, 정적 질량 $M$을 생각하자. 기하학을 수축률 함수 $S(r)$로 기술한다.

\textbf{방사 수축률:}
$$S_r(r) = \sqrt{1 - \frac{2GM}{rc^2}} = \sqrt{1 - \frac{r_s}{r}}$$

\textbf{접선 수축률:}
$$S_\theta(r) = 1$$

\textbf{시간 수축률:}
$$S_t(r) = \sqrt{1 - \frac{2GM}{rc^2}} = \sqrt{1 - \frac{r_s}{r}}$$

\textbf{물리적 해석:}
\begin{itemize}
\item $r > r_s$에서 $S_r < 1$: 방사 거리가 수축됨
\item $S_\theta = 1$: 접선 거리는 불변
\item $S_t < 1$: 시간 간격이 수축됨
\item $r \to \infty$일 때: 모든 $S \to 1$ (질량에서 멀리 떨어지면 수축 없음)
\item $r \to r_s$일 때: $S_r, S_t \to 0$ (사건 지평선에서 완전한 수축)
\end{itemize}

\subsection{슈바르츠실트 계량과의 동등성---그리고 그것이 드러내는 것}

슈바르츠실트 해는(Schwarzschild, 1916; Wald, 1984):

$$ds^2 = -\left(1 - \frac{r_s}{r}\right)c^2 dt^2 + \left(1 - \frac{r_s}{r}\right)^{-1} dr^2 + r^2 d\Omega^2$$

수축 언어로:

$$ds^2 = -(S_t)^2 c^2 dt^2 + (S_r)^{-2} dr^2 + r^2 d\Omega^2$$

여기서 $S_t(r) = S_r(r) = \sqrt{1 - r_s/r}$이다.

\textbf{이는 정확히 같은 계량---하지만 재정식화가 숨겨진 구조를 드러낸다:}

\textbf{표준 형태가 가리는 것:}
\begin{itemize}
\item 인자 $(1 - r_s/r)$이 직관적 의미 없이 나타남
\item 방사 항의 역수 $(1 - r_s/r)^{-1}$이 임의적으로 보임
\item 왜 이런 특정 조합인가? 표준 일반상대성이론은 ``아인슈타인 방정식이 요구하기 때문''이라고 함
\end{itemize}

\textbf{수축 형태가 드러내는 것:}
\begin{itemize}
\item $S_t < 1$: 시간 간격이 압축됨---시계가 더 느리게 진행
\item $S_r < 1$: 방사 공간이 압축됨---자가 더 짧은 거리를 측정
\item 역수 $(S_r)^{-2}$는 명확한 의미를 갖음: 압축된 공간에서 측지선 구조 유지
\item 물리적 해석이 즉각적: 질량이 국소적으로 시공간 측정을 재척도화
\end{itemize}

\textbf{자명하지 않은 통찰:}

슈바르츠실트 기하학에서 $S_t = S_r$이라는 것은 \textbf{물리적 결과}이지 정의적 선택이 아니다. 이는 다음에서 따라온다:
\begin{enumerate}
\item 구형 대칭
\item 정적 배치(에너지 플럭스 없음)
\item 아인슈타인 장 방정식
\end{enumerate}

다른 시공간(예: 우주론)에서는 $S_t \neq S_r$을 발견하며, 시간과 공간에 대한 다른 수축률을 드러낸다. 수축 정식화는 이런 구별을 투명하게 만든다.

\subsection{고전적 예측의 복원}

수축 틀에서 모든 고전적 검증을 유도할 수 있다:

\textbf{중력 시간 팽창:} 반지름 $r_1$과 $r_2$에서 정지한 두 시계는 다음 비율로 진행한다:

$$\frac{\Delta \tau_1}{\Delta \tau_2} = \frac{S_t(r_1)}{S_t(r_2)} = \sqrt{\frac{1 - r_s/r_1}{1 - r_s/r_2}}$$

\textbf{중력 적색편이:} $r_1$에서 방출된 광자는 $r_2$에서 다음 주파수로 수신된다:

$$\frac{\nu_2}{\nu_1} = \frac{S_t(r_2)}{S_t(r_1)}$$

\textbf{근일점 세차:} 타원 궤도의 경우:

$$\Delta \phi = \frac{6\pi GM}{c^2 a(1-e^2)}$$

수성의 경우: $\Delta \phi \approx 43''$ (세기당), 관측과 일치.

\textbf{중력 렌즈:} 질량 $M$로부터 거리 $b$를 지나는 빛의 편향각:

$$\alpha = \frac{4GM}{c^2 b}$$

\subsection{좌표 의존성과 사영좌표계 형식}

수축률 $S(r)$은 특정 좌표에 대해 정의된다. 이를 사영좌표계 형식으로 다룬다.

사영좌표계 장 $e^a_\mu(x)$는 다음을 만족한다:

$$g_{\mu\nu}(x) = \eta_{ab} \, e^a_\mu(x) \, e^b_\nu(x)$$

슈바르츠실트 시공간에 대해:

$$e^{\hat{t}}_\mu = \left(\frac{1}{S_t}, 0, 0, 0\right), \quad e^{\hat{r}}_\mu = \left(0, S_r, 0, 0\right)$$

수축률이 사영좌표계 성분으로 직접 나타나며, 좌표 독립적 기초를 제공한다.

\section{동적 수축으로서의 중력파}

\subsection{정적에서 동적 수축으로}

\textbf{정적 수축:} 정지한 질량은 영구적인 수축 패턴을 만든다.

\textbf{동적 수축:} 가속하는 질량은 시간에 따라 변하는 패턴을 만든다. 이런 변화는 중력파로 속도 $c$로 전파된다.

$z$-방향으로 전파하는 ``+'' 편광 파동의 경우:

$$ds^2 \approx -c^2dt^2 + [1 + h_+(t-z/c)]dx^2 + [1 - h_+(t-z/c)]dy^2 + dz^2$$

\textbf{수축 해석:}

$$S_x(t,z) = 1 + h_+(t-z/c), \quad S_y(t,z) = 1 - h_+(t-z/c)$$

인수 $(t - z/c)$는 수축 교란이 속도 $c$로 전파됨을 보여준다.

\section{텔레평행 중력과의 비교}

텔레평행 중력(Weitzenböck, 1923; Aldrovandi \& Pereira, 2013)과 본 수축 틀은 깊은 연결을 공유하지만, 개념적 초점에서 차이가 있다.

\subsection{수학적 연결}

텔레평행 중력에서 사영좌표계 장 $e^a_\mu(x)$를 도입한다:

$$g_{\mu\nu}(x) = \eta_{ab} \, e^a_\mu(x) \, e^b_\nu(x)$$

슈바르츠실트 시공간에 대해:

$$e^{\hat{t}}_\mu = \left(\frac{1}{S_t}, 0, 0, 0\right), \quad e^{\hat{r}}_\mu = \left(0, S_r, 0, 0\right)$$

\textbf{관찰:} 수축률 $S_i$가 사영좌표계 성분으로 직접 나타난다! 이는 수학적 삼원성을 드러낸다:

\textbf{곡률 정식화(아인슈타인, 1916):}
\begin{itemize}
\item 레비-치비타 접속 사용(비틀림 없음, $T^\lambda_{\mu\nu} = 0$)
\item 중력이 리만 곡률 텐서 $R^\rho_{\sigma\mu\nu}$로 인코딩됨
\item 장 방정식: $G_{\mu\nu} = 8\pi G T_{\mu\nu}/c^4$
\end{itemize}

\textbf{비틀림 정식화(바이첸뵉, 1923):}
\begin{itemize}
\item 바이첸뵉 접속 사용(곡률 없음, $R^\rho_{\sigma\mu\nu} = 0$)
\item 중력이 비틀림 텐서 $T^\lambda_{\mu\nu}$로 인코딩됨
\item 비틀림 스칼라를 통한 동등 장 방정식
\end{itemize}

\textbf{수축 정식화(본 연구):}
\begin{itemize}
\item 사영좌표계 대각 성분 $S_i(x)$를 주요 변수로 사용
\item 중력이 재척도화 함수로 직접 인코딩됨
\item 물리적 해석: 국소 측정 표준이 변형됨
\end{itemize}

\subsection{수학적 동등성에도 불구한 개념적 차이}

\textbf{이 정식화를 독특하게 만드는 것:}

\begin{enumerate}
\item \textbf{텔레평행 중력}은 비틀림을 사용하여 재정식화하지만 접속, 평행 이동, 공변 도함수의 기하학적 언어를 유지한다. 학생들은 여전히 미분기하학 훈련이 필요하다.

\item \textbf{수축 틀}은 측정 재척도화를 사용하여 재정식화한다. 접속도 평행 이동도 없다---단지 ``자가 인자 $S_r$로 줄어들고, 시계가 인자 $S_t$로 느려진다''. 특수상대성이론을 이해하는 학생들에게 접근 가능하다.

\item \textbf{다른 물리적 질문이 나타난다:}
\begin{itemize}
\item 텔레평행: ``중력은 비틀림인가 곡률인가?''
\item 수축: ``측정 재척도화는 어떤 속도로 전파되는가?''
\end{itemize}
\end{enumerate}

두 번째 질문은 표준 일반상대성이론이나 텔레평행 중력(전파 속도 $c$가 기하학적 구조에 내장됨)에서는 정식화할 수 없지만 여기서는 자연스럽게 나타나며, 잠재적으로 새로운 물리를 열 수 있다.

\subsection{상보적 장점}

이 정식화들은 경쟁자가 아니라 상보적 도구다:

\begin{itemize}
\item \textbf{곡률}: 전역 기하학적 성질, 위상, 특이점 정리에 최적
\item \textbf{텔레평행}: 게이지 이론, 스피너 장과 연결에 최적
\item \textbf{수축}: 교육, 명시적 국소성, 잠재적 우주론적 확장에 최적
\end{itemize}

셋 모두 고전 일반상대성이론에 대해 수학적으로 동등하다. 선택은 어떤 물리적 통찰을 강조하고 싶은가에 달려 있다.

\section{교육적 함의}

\subsection{재정식화의 철학적 지위}

수축 틀은 새로운 이론인가 단지 새로운 표기법인가?

\textbf{저의 입장:} 실용적 입장을 취한다. 수축과 곡률 틀은 고전 수준에서 경험적으로 동등하다. 언어는 사고와 연구 방향을 형성할 수 있다.

\textbf{인식론적 겸손:} ``중력의 진정한 본성''을 발견했다고 주장하지 않는다. 한 수학적 틀이 다른 틀과 같은 물리적 내용을 표현할 수 있음을 보였다.

\subsection{교육적 가치}

수축 틀이 중력을 가르치는 방법을 개선할 수 있는가? 이는 답이 없는 경험적 질문이며 교육 연구로 검증해야 한다.

\textbf{잠재적 이점(검증할 가설):}

\begin{enumerate}
\item \textbf{직관적 시각화}: ``수축된 공간''이 ``휘어진 4차원 다양체''보다 상상하기 쉬울 수 있다
\item \textbf{명시적 국소성}: 각 질량이 바로 주변에만 작용한다는 강조
\item \textbf{통일된 처리}: 질량이 있든 없든 모두 변형된 기하학에서 측지선을 따른다
\end{enumerate}

\textbf{중요한 주의사항:} 이런 이점은 이론적이다---통제된 연구를 수행하지 않았다.

\subsection{한계와 주의사항}

\textbf{한계 1: 엄밀한 증명의 범위}

수축 정식화가 일반상대성이론과 수학적으로 동등함을 엄밀하게 증명한 것은:
\begin{itemize}
\item 정적, 구형 대칭 시공간(슈바르츠실트)
\item 약장 중력파
\item 다체계(약장 근사)
\end{itemize}

\textbf{한계 2: 좌표 의존성}

사영좌표계 형식으로 다루었으며, 좌표 독립적 기초를 제공한다. 그러나 수축률 $S_i$의 구체적 값은 여전히 좌표 선택에 의존한다.

\textbf{한계 3: 정적 시공간에서 새로운 예측 없음}

고전 수준에서 정적 시공간에 대해, 이는 재정식화지 새로운 이론이 아니다. 모든 예측이 표준 일반상대성이론과 동일하다.

\textbf{한계 4: 우주론적 확장의 추측적 성격}

7절에서 제시한 우주론적 아이디어(황금 지수, DDR 위반, 허블 텐션 등)는 \textbf{엄밀하게 증명되지 않았다}. 이는 향후 연구를 위한 동기 부여와 탐구 방향이지, 확립된 결과가 아니다. 신중한 이론적 발전과 관측 검증이 필수적이다.

\section{우주론으로의 확장 가능성}

\subsection{동기: 재정식화가 새로운 물리를 시사할 때}

수축 틀이 정적 시공간에서 일반상대성이론과 수학적으로 동등하지만, ``시공간 곡률''에서 ``국소 수축''으로의 개념적 전환은 표준 일반상대성이론에서 정식화하기 어려운 질문들을 자연스럽게 제기한다.

특히, 수축 그림은 \textbf{조정된 기하학적 전파}보다는 \textbf{독립적 국소 작용}을 강조한다. 이는 흥미로운 질문을 제기한다: 각 시공간 점이 국소 물질 밀도에 기반하여 독립적으로 수축한다면, 먼 점들 사이의 동기화가 정확해야 하는가, 아니면 근사적일 수 있는가?

표준 일반상대성이론은 이 질문을 다룰 수 없다---빛의 속도 $c$가 정확한 제약으로 기하학적 구조에 내장되어 있다. 그러나 수축 틀에서, $S_i(x,t)$가 동역학적 장으로 취급될 때, 국소 수축 사이의 ``조정 속도''는 기하학적 공리보다는 물리적 파라미터가 된다.

이러한 개념적 열림은, 현대 우주론에서 지속적인 관측 긴장과 결합하여, 수축 관점이 우주론적 규모에서 수정을 드러내는지 탐구하도록 동기를 부여한다.

\subsection{황금비 스케일링: 정보 이론적 연결}

시공간 수축이 최적의 정보 처리 메커니즘을 나타낸다면, 차원 분석은 자연스러운 스케일링 관계를 시사한다.

\textbf{차원적 불일치를 고려하라:}
\begin{itemize}
\item 3차원 공간 부피는 $V \sim L^3$로 스케일
\item 4차원 시공간 곡률은 $R \sim L^{-2}$를 포함 (리만 텐서에서)
\item 에너지 밀도 $\rho$는 차원 $[M L^{-3}]$를 가짐
\end{itemize}

공간 수축(3D)을 시공간 곡률(4D)에 연결하기 위해, 공간과 시간 수축 사이의 최적 정보 전달을 요구하는 것에서 무차원 스케일링 비율이 나타난다.

\textbf{황금비 $\varphi = \frac{1+\sqrt{5}}{2} \approx 1.618$은 다음을 최적화하는 시스템에서 자연스럽게 나타난다:}
\begin{enumerate}
\item 효율적 패킹 (자연의 피보나치 나선)
\item 정보 압축 (연분수 수렴)
\item 차원 간 자기 유사 스케일링
\end{enumerate}

\textbf{제안된 스케일링 가정:}

유효 수축 전파가 차원 최적화와 관련된 인자만큼 빛의 속도와 다르다면:

$$c_{\text{eff}} = c \cdot \varphi^{\alpha}$$

여기서 $\alpha$는 공간(3D)과 시공간(4D) 구조 사이의 차원적 불일치로 결정된다. $\alpha \sim 0.1$에 대해, 이는 다음을 준다:

$$\frac{c_{\text{eff}}}{c} \approx 1.05$$

이 5% 편차는 태양계 규모에서는 무시할 수 있지만 우주론적 거리에 걸쳐 상당히 누적된다.

\subsection{거리 이중성 관계: 구체적 관측 검증}

표준 우주론에서 거리 이중성 관계(DDR)는 광도 거리 $D_L$과 각지름 거리 $D_A$가 다음을 만족한다고 가정한다:

$$\frac{D_L}{D_A} = (1+z)^2$$

이는 일반상대성이론에서 위반이 예상되지 않으면서 정확히 성립한다.

\textbf{수축 유도 DDR 위반:}

공간 수축이 큰 규모에서 약간의 비등방성을 나타내면(변형된 동기화로 인해), 관계는 다음과 같이 된다:

$$\frac{D_L}{D_A} = (1+z)^2 [1 + \eta(z)]$$

여기서 위반 파라미터 $\eta(z)$는 누적된 비동기화에 의존한다.

\textbf{정량적 예측:}

수축 조정이 $\varphi$-스케일링을 나타내면, 차원 분석은 다음을 시사한다:

$$\eta(z) \approx \eta_0 \cdot \frac{z^2}{(1+z)^3}$$

$z \sim 2$에서 최대 위반이 나타나고 크기는 $|\eta(z=2)| \sim 0.015$에서 $0.02$ (1.5--2\%).

\textbf{관측 현황:}

Ia형 초신성과 강한 중력 렌즈를 결합한 최근 분석은 $z \sim 1.5$--$2.5$에서 2--3$\sigma$ 수준의 DDR 위반 힌트를 보고한다 \citep{liao2016}. 2023년 발사된 Euclid 우주 망원경은 $0.5 < z < 3$에 걸쳐 $<0.5\%$ 정밀도로 $\eta(z)$에 대한 결정적인 제약을 제공할 것이다.

\subsection{허블 텐션: 국소 수축 기울기}

허블 상수 측정은 지속적인 불일치를 보인다:
\begin{itemize}
\item 국소 (Cepheids + SNe Ia): $H_0 = 73.0 \pm 1.0$ km/s/Mpc \citep{riess2022}
\item 초기 우주 (CMB): $H_0 = 67.4 \pm 0.5$ km/s/Mpc \citep{planck2018}
\end{itemize}

이 $5.6\sigma$ 긴장은 $\Lambda$CDM 내에서 해결에 저항해 왔다.

\textbf{수축 해석:}

우리가 국소 저밀도 영역(KBC void, $r \sim 100$ Mpc에서 $\delta \rho / \rho \sim -0.15$)에 거주한다면, 증가된 국소 공간 수축이 먼 은하들을 더 빠르게 후퇴하는 것처럼 보이게 만든다.

\textbf{정량적 모델:}

국소 수축률 증가:

$$S_{\text{local}} = S_{\text{global}} \cdot \left(1 + \frac{\delta\rho}{\rho_{\text{crit}}} \cdot f_{\varphi}\right)$$

여기서 $f_{\varphi} \approx 0.08$은 $\varphi$-스케일링 결합 강도를 나타낸다.

이는 겉보기 국소 허블 상수를 만든다:

$$H_{0,\text{local}} = H_{0,\text{global}} \cdot \left(1 + 0.08 \times 0.15\right) \approx 1.09 \, H_{0,\text{global}}$$

$H_{0,\text{global}} = 67$ km/s/Mpc에 대해, 이는 $H_{0,\text{local}} \approx 73$ km/s/Mpc를 예측하며, 관측과 일치한다.

\subsection{구조 형성: 뭉침에 대한 수축 저항}

물질 뭉침 진폭 $\sigma_8$은 초기 우주와 후기 시간 측정 사이의 긴장을 보인다:
\begin{itemize}
\item Planck CMB (초기): $\sigma_8 = 0.830 \pm 0.013$
\item 약한 렌즈 (후기): $\sigma_8 = 0.759 \pm 0.025$ \citep{heymans2021}
\end{itemize}

\textbf{수축 저항 메커니즘:}

공간 수축은 물질 뭉침에 동역학적 저항을 제공한다. 과밀도가 형성될 때, 증가된 국소 수축이 추가 붕괴에 반대하며, 후기 시간에 성장률을 효과적으로 감소시킨다.

\textbf{수정된 성장 방정식:}

선형 성장 인자 $D(a)$가 저항 항을 얻는다:

$$\frac{d^2 D}{da^2} + \left[\frac{2}{a} + \frac{d\ln H}{da}\right]\frac{dD}{da} = \frac{3\Omega_m}{2a^3} D - \gamma_{\varphi} \frac{dD}{da}$$

여기서 $\gamma_{\varphi} \approx 0.05$는 수축 유도 저항을 나타낸다.

이는 후기 시간 뭉침을 $\sim 5$--$8\%$ 억제하며, 암흑에너지나 중성미자 질량을 수정하지 않고 자연스럽게 $\sigma_8$을 낮춘다.


\subsection{검증 가능성과 반증 가능성}

이러한 우주론적 확장은 구체적이고 반증 가능한 예측을 한다:

\begin{enumerate}
\item \textbf{DDR 위반}: Euclid는 2028년까지 $|\eta(z)|$를 $<0.5\%$로 제한할 것이다. $\eta(z=2) > 0.01$이 확인되면, 이는 수축 비등방성을 지지한다.

\item \textbf{허블 기울기}: 증가하는 거리에서의 국소 $H_0$ 측정이 $\sim 200$ Mpc에 걸쳐 73에서 67 km/s/Mpc로의 매끄러운 전환을 보이면, 이는 국소 수축 기울기를 확인한다.

\item \textbf{성장률}: 다가오는 DESI와 Vera Rubin Observatory의 $f\sigma_8(z)$ 측정은 성장 억제가 $\gamma_{\varphi} \sim 0.05$ 예측과 일치하는지 검증할 것이다.

\item \textbf{$\varphi$-상관관계}: 세 가지 효과(DDR, Hubble, $\sigma_8$)가 모두 일관된 $\alpha \sim 0.08$--$0.10$을 가진 공통 스케일링 인자 $\varphi^{\alpha}$로 추적되면, 이는 수축 틀을 강력하게 지지한다.
\end{enumerate}

\textbf{중요한 주의사항:}

이러한 확장은 매우 추측적이다. $\varphi$-스케일링은 차원 분석과 정보 이론적 논증으로 동기 부여되지만, 제1원리로부터의 엄밀한 유도가 부족하다. 우주론적 긴장에 대한 대안적 설명이 존재한다(수정 중력, 초기 암흑에너지, 체계적 오차).

이 절의 가치는 이러한 효과가 입증되었다고 주장하는 데 있지 않고, 수축 재정식화가 표준 일반상대성이론이 쉽게 동기 부여하지 않는 구체적이고 검증 가능한 수정을 자연스럽게 제안한다는 것을 보여주는 데 있다.

\section{결론}

\subsection{요약}

곡률 대신 \textbf{국소 시공간 수축}을 사용하여 일반상대성이론의 재해석을 제안했다:

\begin{enumerate}
\item \textbf{개념적 틀:} 각 질량은 주변에서 시공간을 독립적으로 수축시킨다. 모든 입자는 이 변형된 기하학에서 측지선을 따른다.

\item \textbf{수학적 정식화:} 수축률은 슈바르츠실트 계량을 정확히 재현한다. 모든 고전적 예측을 복원한다.

\item \textbf{중력파:} 동적 수축은 속도 $c$로 전파된다.

\item \textbf{비교:} 사영좌표계 형식을 통한 텔레평행 중력과의 명시적 수학적 연결.
\end{enumerate}

\subsection{주요 기여}

\textbf{개념적 기여:} 일반상대성이론의 기하학적 내용을 \textbf{전역 곡률}이 아닌 \textbf{측정 표준의 국소 변화}로 표현할 수 있음을 보였다. 이 재정식화는 정적 시공간에서 수학적으로 동등하지만, 우주론으로 확장할 때 다른 질문과 가능성을 자연스럽게 제기한다.

\textbf{수학적 기여:}
\begin{itemize}
\item 원격 작용의 형식적 정의
\item 수축 정식화가 국소성을 더 명시적으로 만든다는 증명
\item ``수축'' 용어의 명확화
\item 사영좌표계 형식을 통한 명시적 처리
\item 텔레평행 중력과의 연결
\end{itemize}

\textbf{교육적 기여:} 수축 틀은 학생들에게 더 직관적인 진입점을 제공할 수 있다. 리만 기하학의 무거운 기술적 부담 없이도 중력의 핵심 아이디어---국소성, 기하학적 변형, 측지선 운동---를 전달할 수 있다.

\textbf{우주론적 기여:} 7절은 수축 관점이 자연스럽게 검증 가능한 확장을 시사함을 보여준다:
\begin{itemize}
\item $z \sim 2$에서 크기 $\sim 1.7\%$인 DDR 위반
\item 국소 수축 기울기를 통한 허블 텐션 해결
\item 수축 저항을 통한 구조 형성 억제
\item 세 가지 효과를 모두 연결하는 통일된 $\varphi$-스케일링
\end{itemize}

이러한 예측은 관측적 확인 또는 반증을 기다린다.

\subsection{향후 연구}

수축 틀은 여러 방향으로 자연스러운 확장을 시사한다:

\textbf{우주론적 응용:}

수축 틀을 정적 시공간을 넘어 확장하는 가장 흥미로운 방향은 우주론에 수축 동역학을 적용하는 것이다. 핵심 질문: 시공간 수축이 전파된 상호작용보다는 진정으로 독립적인 국소 작용을 나타낸다면, 수축의 효과적인 ``동기화 속도''가 빛의 속도와 다를 수 있는가?

표준 일반상대성이론에서는 이 질문을 정식화할 수 없다---속도 $c$가 기하학적 구조에 내장되어 있다. 그러나 수축 그림에서, 각 점이 독립적으로 수정하는 경우, 먼 수축 사이의 조정은 기하학적 공리보다는 동역학적 질문이 된다.

우주론에서 최근의 지속적인 긴장---특히 국소와 초기 우주 측정 사이의 허블 상수 불일치, 그리고 다른 관측 시기 사이의 물질 뭉침 진폭 긴장---은 표준 우주론 모델 내에서 해결에 저항해 왔다. 이러한 긴장은 국소 수축 사이의 완벽한 빛의 속도 조정에 대한 우리의 가정이 정확하기보다는 근사적임을 나타낼 수 있다.

수축 동역학이 우주론적 규모에서 수정된 조정 속도를 허용한다면, 이는 다음을 할 것이다:
\begin{enumerate}
\item 허블 파라미터를 특정하고 검증 가능한 방식으로 적색편이 의존적으로 만듦
\item 변경된 중력 뭉침을 통해 구조 형성 예측 수정
\item 끈 이론의 k-essence와 DBI 틀에 연결
\item 국소 검증(태양계, 중력파)과 완전히 양립 가능하게 유지
\end{enumerate}

이 방향은 신중한 이론적 발전과 관측 검증이 필요하다. 미래의 중력파 관측과 대규모 구조 탐사는 수축 관점이 우주론적 규모에서 새로운 물리를 드러내는지에 대한 결정적인 제약을 제공할 것이다.

\textbf{다른 방향:}
\begin{itemize}
\item 회전하는 블랙홀(Kerr 계량 재정식화)
\item 수축의 양자 측면
\item 경험적 연구를 통한 교육적 효과성
\item 창발적 중력 및 홀로그래픽 원리와의 연결
\end{itemize}


\textbf{핵심 철학적 질문:}

궁극적으로, 문제는 수축이나 곡률이 ``옳은가''가 아니다. 정적 시공간에서 둘 다 같은 물리를 정확히 기술한다. 진짜 질문은: \textit{어떤 정식화가 우리를 새로운 통찰로 이끄는가?}

역사는 재정식화가 강력할 수 있음을 보여준다. 뉴턴의 $F = ma$와 라그랑주의 최소 작용 원리는 고전 역학에서 동등하지만, 라그랑주 형식이 양자 역학으로의 길을 열었다. 마찬가지로, 수축 관점이 우주론적 긴장을 해결하는 개념적 경로를 열 수 있는지는 열린 질문이다.

답은 이론적 선호가 아니라 자연 자체---다가오는 정밀 관측 데이터---에서 와야 한다.

\bibliographystyle{plainnat}
\begin{thebibliography}{99}

\bibitem{einstein1916} Einstein, A. (1916). Die Grundlage der allgemeinen Relativitätstheorie. \emph{Annalen der Physik}, 49, 769--822.

\bibitem{schwarzschild1916} Schwarzschild, K. (1916). Über das Gravitationsfeld eines Massenpunktes nach der Einsteinschen Theorie. \emph{Sitzungsberichte der Königlich Preussischen Akademie der Wissenschaften}, 189--196.

\bibitem{mtw1973} Misner, C. W., Thorne, K. S., \& Wheeler, J. A. (1973). \emph{Gravitation}. W. H. Freeman.

\bibitem{wald1984} Wald, R. M. (1984). \emph{General Relativity}. University of Chicago Press.

\bibitem{carroll2004} Carroll, S. M. (2004). \emph{Spacetime and Geometry: An Introduction to General Relativity}. Addison-Wesley.

\bibitem{abbott2016} Abbott, B. P., et al. (2016). Observation of gravitational waves from a binary black hole merger. \emph{Physical Review Letters}, 116, 061102.

\bibitem{weitzenboeck1923} Weitzenböck, R. (1923). \emph{Invariantentheorie}. Noordhoff.

\bibitem{aldrovandi2013} Aldrovandi, R., \& Pereira, J. G. (2013). \emph{Teleparallel Gravity: An Introduction}. Springer.

\bibitem{barbour2012} Barbour, J. (2012). Shape dynamics. In \emph{Quantum Field Theory and Gravity} (pp. 257--297). Birkhäuser.

\bibitem{clemence1947} Clemence, G. M. (1947). The relativity effect in planetary motions. \emph{Reviews of Modern Physics}, 19, 361--364.

\bibitem{abramowicz1997} Abramowicz, M. A., \& Lasota, J.-P. (1997). A medium where gravity acts as optics. \emph{General Relativity and Gravitation}, 29, 1377--1391.

\bibitem{liao2016} Liao, K., et al. (2016). Constraints on cosmic opacity and beyond the standard model physics from distance duality relation. \emph{The Astrophysical Journal}, 822, 74.

\bibitem{riess2022} Riess, A. G., et al. (2022). A comprehensive measurement of the local value of the Hubble constant with 1 km/s/Mpc uncertainty from the Hubble Space Telescope and the SH0ES Team. \emph{The Astrophysical Journal Letters}, 934, L7.

\bibitem{planck2018} Planck Collaboration (2018). Planck 2018 results. VI. Cosmological parameters. \emph{Astronomy \& Astrophysics}, 641, A6.

\bibitem{heymans2021} Heymans, C., et al. (2021). KiDS-1000 cosmology: Multi-probe weak gravitational lensing and spectroscopic galaxy clustering constraints. \emph{Astronomy \& Astrophysics}, 646, A140.

\end{thebibliography}

\appendix

\section{상세한 측지선 계산}

입자의 라그랑주안은:

$$\mathcal{L} = \frac{1}{2}\left[-(S_t)^2 c^2 \dot{t}^2 + (S_r)^{-2}\dot{r}^2 + r^2(\dot{\theta}^2 + \sin^2\theta \, \dot{\phi}^2)\right]$$

여기서 점은 $d/d\tau$를 나타내고 $S_t = S_r = \sqrt{1 - r_s/r}$이다.

시간 병진 불변성에서:

$$E = (S_t)^2 c^2 \dot{t}$$

회전 대칭에서:

$$L = r^2 \sin^2\theta \, \dot{\phi}$$

\section{근일점 세차 유도}

$u = 1/r$로 하여 궤도 방정식에서 출발한다:

$$\frac{d^2 u}{d\phi^2} + u = \frac{GM}{L^2} + 3GMu^2$$

섭동으로 풀면, 한 궤도 후 근일점은 다음만큼 전진한다:

$$\Delta\phi_{\text{prec}} = \frac{6\pi GM}{c^2 a(1 - e^2)}$$

수성의 경우: $\Delta\phi \approx 43''$ (세기당).

\section{빛 편향 계산}

광자에 대해 $ds^2 = 0$:

$$0 = -(S_t)^2 c^2 dt^2 + (S_r)^{-2}dr^2 + r^2 d\phi^2$$

약장 근사에서:

$$\alpha = \frac{4GM}{c^2 b}$$

태양의 경우 $b = R_{\odot}$에서: $\alpha \approx 1.75''$

\section{표기법과 관례}

\textbf{계량 부호:} $(-,+,+,+)$

\textbf{단위:} SI 단위
\begin{itemize}
\item $c = 2.998 \times 10^8$ m/s
\item $G = 6.674 \times 10^{-11}$ m³/(kg·s²)
\end{itemize}

\textbf{슈바르츠실트 반지름:} $r_s = \frac{2GM}{c^2}$

태양의 경우: $r_s \approx 2.95$ km

\section*{감사의 말}

교육적 대안을 탐구하기 위해 독립적으로 수행되었다. 모든 오류는 저자의 책임이다.

\end{document}
